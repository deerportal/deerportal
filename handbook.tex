\documentclass[12pt,a4paper]{article}
\usepackage[utf8]{inputenc}
\usepackage[T1]{fontenc}
\usepackage{geometry}
\usepackage{graphicx}
\usepackage{hyperref}
\usepackage{titling}
\usepackage{fancyhdr}
\usepackage{xcolor}
\usepackage{enumitem}
\usepackage{tabularx}
\usepackage{booktabs}

\geometry{margin=2.5cm}
\title{Deer Portal \\ \large Game Handbook \& Card Reference}
\date{\today}
\author{Rafał Zawadzki \\ \small bluszcz@gmail.com}

% Custom colors for elements
\definecolor{waterblue}{RGB}{70,130,180}
\definecolor{earthgreen}{RGB}{34,139,34}
\definecolor{firered}{RGB}{220,20,60}
\definecolor{airyellow}{RGB}{255,215,0}

% Header and footer setup
\pagestyle{fancy}
\fancyhf{}
\rhead{Deer Portal Handbook}
\lhead{\leftmark}
\cfoot{\thepage}

\renewcommand\maketitlehooka{\null\mbox{}\vfill}
\renewcommand\maketitlehookd{\vfill\null}

\begin{document}
\maketitle
\newpage

\tableofcontents
\newpage

\section{Introduction}

Deer Portal is a multiplayer board game driven by the four classical elements, designed for 0-4 players. The game takes place in an ancient world where the Almighty Deer God protects all compassionate creatures.

\subsection{Game Objective}
Transform yourself into a \textbf{Deer Lesser God} by reaching the \textbf{Hoof Portal} and collecting the most diamonds.

\subsection{Game Philosophy}
The game was inspired during a Buddhist journey through Japan, specifically after visiting Tōdai-ji temple in Nara. The mission is to spread the word about the Deer through this strategic gameplay.

\section{Game Setup}

\subsection{Classical Elements}
The gameboard is divided into four areas, each controlled by one classical element:

\begin{itemize}
    \item \textcolor{waterblue}{\textbf{Water}} (Blue) - Represents flow, adaptation, and change
    \item \textcolor{earthgreen}{\textbf{Earth}} (Green) - Represents stability, growth, and foundation  
    \item \textcolor{firered}{\textbf{Fire}} (Red) - Represents energy, transformation, and passion
    \item \textcolor{airyellow}{\textbf{Air}} (Yellow) - Represents freedom, intellect, and movement
\end{itemize}

\subsection{Players}
Each of the four players is mentored by one classical element and can:
\begin{itemize}
    \item Move around their designated board area
    \item Collect runes and diamonds
    \item Meditate to receive help from their patron element
    \item Reach the Deer Portal for transformation
\end{itemize}

Players can be controlled by humans or computers, allowing for various gameplay configurations.

\subsection{Turn Structure}
Each active player (who hasn't reached the portal yet, nor is frozen) performs two actions during their turn:

\begin{enumerate}
    \item \textbf{Roll the Dice} - Click on the dice symbol in the bottom right area
    \item \textbf{Choose Movement} - Click on the corresponding area on your board to move your character
\end{enumerate}

\subsection{Field Effects}
After moving your character, depending on the destination field you may:

\begin{itemize}
    \item \textbf{Collect Diamonds} - If there was one on the field (+1 cash)
    \item \textbf{Activate Cards} - If a rune was on the field (execute card effect)
    \item \textbf{Meditate} - If the field was your starting place (regenerate your area)
    \item \textbf{Enter Portal} - Move to the Deer God kingdom (end game trigger)
\end{itemize}

\section{Card System}

The heart of Deer Portal's strategy lies in its sophisticated card system. Each element maintains a deck of 32 cards that provide powerful interactions between players.

\subsection{Card Distribution}
\begin{center}
\begin{tabular}{|c|c|c|}
\hline
\textbf{Card Type} & \textbf{Quantity per Element} & \textbf{Total in Game} \\
\hline
Stop Cards & 8 & 32 \\
Remove Cards & 8 & 32 \\
Diamond Cards & 8 & 32 \\
Diamond x2 Cards & 8 & 32 \\
\hline
\textbf{Total per Element} & \textbf{32} & \textbf{128} \\
\hline
\end{tabular}
\end{center}

\subsection{Card Activation Rules}
\begin{enumerate}
    \item \textbf{Own Element Runes}: When landing on your own element's rune, the card is discarded and the next card is revealed
    \item \textbf{Other Element Runes}: When landing on another element's rune, the card effect executes against that element's area
    \item \textbf{Targeting Restriction}: You cannot use cards against your own element area
    \item \textbf{Deck Progression}: After each card use, the deck advances to the next card
    \item \textbf{Deck Exhaustion}: When a deck runs out, that element pile becomes inactive
\end{enumerate}

\section{Card Types \& Effects}

\subsection{Stop Cards}
\textbf{Primary Effect}: Freeze target player for one complete turn

\textbf{Strategic Use}:
\begin{itemize}
    \item Disrupt leading players during critical moments
    \item Prevent opponents from reaching the portal
    \item Buy time to collect valuable resources
    \item Most effective during mid-to-late game phases
\end{itemize}

\textbf{Visual Design}: Each element's stop card features a distinctive yellow prohibition symbol overlaid on the element's thematic background.

\begin{figure}[h]
\centering
\begin{minipage}{0.2\textwidth}
    \centering
    \includegraphics[width=\textwidth]{assets/img/cards/card-water-stop.small.png}
    \caption*{\textcolor{waterblue}{Water Stop}}
\end{minipage}
\hfill
\begin{minipage}{0.2\textwidth}
    \centering
    \includegraphics[width=\textwidth]{assets/img/cards/card-earth-stop.small.png}
    \caption*{\textcolor{earthgreen}{Earth Stop}}
\end{minipage}
\hfill
\begin{minipage}{0.2\textwidth}
    \centering
    \includegraphics[width=\textwidth]{assets/img/cards/card-fire-stop.small.png}
    \caption*{\textcolor{firered}{Fire Stop}}
\end{minipage}
\hfill
\begin{minipage}{0.2\textwidth}
    \centering
    \includegraphics[width=\textwidth]{assets/img/cards/card-air-stop.small.png}
    \caption*{\textcolor{airyellow}{Air Stop}}
\end{minipage}
\end{figure}

\subsection{Remove Cards}
\textbf{Primary Effect}: Eliminate a random diamond or card from target element area

\textbf{Strategic Use}:
\begin{itemize}
    \item Reduce opponent resource accumulation
    \item Eliminate high-value targets before opponents can collect them
    \item Tactical disruption of opponent strategies
    \item Most effective when target areas have valuable visible resources
\end{itemize}

\textbf{Visual Design}: Features a crossed-out diamond symbol on each element's themed background, representing resource elimination.

\begin{figure}[h]
\centering
\begin{minipage}{0.2\textwidth}
    \centering
    \includegraphics[width=\textwidth]{assets/img/cards/card-water-remove-card.small.png}
    \caption*{\textcolor{waterblue}{Water Remove}}
\end{minipage}
\hfill
\begin{minipage}{0.2\textwidth}
    \centering
    \includegraphics[width=\textwidth]{assets/img/cards/card-earth-remove-card.small.png}
    \caption*{\textcolor{earthgreen}{Earth Remove}}
\end{minipage}
\hfill
\begin{minipage}{0.2\textwidth}
    \centering
    \includegraphics[width=\textwidth]{assets/img/cards/card-fire-remove-card.small.png}
    \caption*{\textcolor{firered}{Fire Remove}}
\end{minipage}
\hfill
\begin{minipage}{0.2\textwidth}
    \centering
    \includegraphics[width=\textwidth]{assets/img/cards/card-air-remove-card.small.png}
    \caption*{\textcolor{airyellow}{Air Remove}}
\end{minipage}
\end{figure}

\subsection{Diamond Cards}
\textbf{Primary Effect}: Steal one diamond from target area + award 1 cash to current player

\textbf{Strategic Use}:
\begin{itemize}
    \item Direct resource acquisition from opponents
    \item Build economic advantage throughout the game
    \item Steady progression toward victory condition
    \item Effective throughout all game phases
\end{itemize}

\textbf{Visual Design}: Displays a single, prominently featured diamond on each element's background.

\begin{figure}[h]
\centering
\begin{minipage}{0.2\textwidth}
    \centering
    \includegraphics[width=\textwidth]{assets/img/cards/card-water-diam.small.png}
    \caption*{\textcolor{waterblue}{Water Diamond}}
\end{minipage}
\hfill
\begin{minipage}{0.2\textwidth}
    \centering
    \includegraphics[width=\textwidth]{assets/img/cards/card-earth-diam.small.png}
    \caption*{\textcolor{earthgreen}{Earth Diamond}}
\end{minipage}
\hfill
\begin{minipage}{0.2\textwidth}
    \centering
    \includegraphics[width=\textwidth]{assets/img/cards/card-fire-diam.small.png}
    \caption*{\textcolor{firered}{Fire Diamond}}
\end{minipage}
\hfill
\begin{minipage}{0.2\textwidth}
    \centering
    \includegraphics[width=\textwidth]{assets/img/cards/card-air-diam.small.png}
    \caption*{\textcolor{airyellow}{Air Diamond}}
\end{minipage}
\end{figure}

\subsection{Diamond x2 Cards}
\textbf{Primary Effect}: Steal two diamonds from target area + award 2 cash to current player

\textbf{Strategic Use}:
\begin{itemize}
    \item Maximum resource gain per single card activation
    \item Game-changing potential in close matches
    \item Priority target for strategic timing
    \item Most valuable card type in the deck
\end{itemize}

\textbf{Visual Design}: Features two diamonds prominently displayed, often with enhanced visual effects representing their increased value.

\begin{figure}[h]
\centering
\begin{minipage}{0.2\textwidth}
    \centering
    \includegraphics[width=\textwidth]{assets/img/cards/card-water-2-diam.small.png}
    \caption*{\textcolor{waterblue}{Water Diamond x2}}
\end{minipage}
\hfill
\begin{minipage}{0.2\textwidth}
    \centering
    \includegraphics[width=\textwidth]{assets/img/cards/card-earth-2-diam.small.png}
    \caption*{\textcolor{earthgreen}{Earth Diamond x2}}
\end{minipage}
\hfill
\begin{minipage}{0.2\textwidth}
    \centering
    \includegraphics[width=\textwidth]{assets/img/cards/card-fire-2-diam.small.png}
    \caption*{\textcolor{firered}{Fire Diamond x2}}
\end{minipage}
\hfill
\begin{minipage}{0.2\textwidth}
    \centering
    \includegraphics[width=\textwidth]{assets/img/cards/card-air-2-diam.small.png}
    \caption*{\textcolor{airyellow}{Air Diamond x2}}
\end{minipage}
\end{figure}

\section{Element-Specific Visual Themes}

\subsection{\textcolor{waterblue}{Water Element Cards}}
Visual characteristics include flowing water textures, blue gradients, and aquatic motifs that convey fluidity and adaptability.

\subsection{\textcolor{earthgreen}{Earth Element Cards}}
Feature natural textures, green earth tones, and organic patterns representing stability and growth.

\subsection{\textcolor{firered}{Fire Element Cards}}
Incorporate flame effects, warm red colors, and dynamic energy patterns symbolizing transformation and passion.

\subsection{\textcolor{airyellow}{Air Element Cards}}
Display light, airy backgrounds with yellow/white color schemes and flowing patterns representing freedom and movement.

\section{Advanced Strategy}

\subsection{Card Priority System}
\begin{enumerate}
    \item \textbf{Diamond x2 Cards} - Highest strategic value
    \item \textbf{Diamond Cards} - Reliable resource acquisition  
    \item \textbf{Stop Cards} - Tactical disruption tool
    \item \textbf{Remove Cards} - Situational utility option
\end{enumerate}

\subsection{Timing Strategies}
\begin{itemize}
    \item \textbf{Early Game}: Focus on diamond cards for resource building
    \item \textbf{Mid Game}: Deploy stop cards against leading players
    \item \textbf{Late Game}: Use remove cards to eliminate threats before portal entry
\end{itemize}

\subsection{Element Targeting}
Consider these factors when choosing targets:
\begin{itemize}
    \item Visible resource density in target areas
    \item Strategic position of controlling players
    \item Current game state and remaining turns
    \item Deck status of different elements
\end{itemize}

\section{Game Mechanics Integration}

\subsection{Meditation System}
Returning to your starting position with an exact dice roll triggers meditation, which:
\begin{itemize}
    \item Regenerates all diamonds and cards in your area
    \item Requires precise dice roll calculation
    \item Provides strategic reset opportunity
    \item Essential for long-term resource management
\end{itemize}

\subsection{Deck Exhaustion Effects}
When element decks become empty:
\begin{itemize}
    \item Deck becomes inactive (no more cards can be drawn)
    \item All remaining resources of that element are removed from the board
    \item Element becomes strategically less valuable
    \item Affects long-term game balance
\end{itemize}

\subsection{Deer Mode}
When the first player reaches the portal:
\begin{itemize}
    \item \textbf{Deer Mode} begins (4 rounds - one turn per player)
    \item First player gets bonus diamonds for reaching first
    \item All remaining players must reach portal before Deer Mode ends
    \item Players who don't reach portal in time are eliminated
\end{itemize}

\section{Victory Conditions}

\subsection{Primary Victory}
The player with the most diamonds when all players reach the portal (or deer mode ends) wins the game.

\subsection{Tiebreaker}
In case of diamond count ties, victory goes to the player who reached the portal first.

\subsection{Transformation}
- \textbf{Winner}: Transforms into a Deer God
- \textbf{Survivors}: Become devoted monks spreading the philosophy  
- \textbf{Eliminated}: Become the foundation of the new philosophy

\section{Advanced Tips}

\subsection{Resource Management}
\begin{itemize}
    \item Balance immediate gains vs. long-term strategy
    \item Monitor opponent diamond counts
    \item Use meditation strategically to refresh your area
    \item Time your portal entry carefully
\end{itemize}

\subsection{Card Timing}
\begin{itemize}
    \item Save powerful cards for crucial moments
    \item Watch for opponent weaknesses
    \item Don't waste cards on empty areas
    \item Consider deck status when planning moves
\end{itemize}

\subsection{Movement Strategy}
\begin{itemize}
    \item Plan your dice rolls and movement paths
    \item Position for multiple collection opportunities
    \item Avoid predictable patterns
    \item Use exact movement for meditation when needed
\end{itemize}

\subsection{Endgame Planning}
\begin{itemize}
    \item Monitor Deer Mode countdown
    \item Ensure portal access before elimination
    \item Make final resource grabs before time runs out
    \item Position for tiebreaker advantages
\end{itemize}

\section{Conclusion}

The Deer Portal card system provides deep strategic gameplay while maintaining thematic coherence with the game's spiritual and elemental foundations. Mastery of card timing, element targeting, and resource management separates novice players from those worthy of transformation into Deer Gods.

May your journey through the elements lead to enlightenment and victory in the sacred realm of the Almighty Deer.

\end{document}